\subsubsection{Blind Assignment for Blockchain Extension (BABE)}\label{sec:babe}

In Polkadot, we produce relay chain blocks using our Blind Assignment for Blockchain Extension protocol, abbreviated BABE. BABE assigns validators randomly to block production slots using  the randomness generated with blocks. These assignments are completely private until the assigned parties produce their blocks. Therefore, we use ``Blind Assignment'' in the protocol name. BABE is similar to Ouroboros Praos \cite{Praos} with some significant differences in the chain selection rule and timing assumptions.

In BABE, we may have slots without any assignment which we call empty slot. In order to fill the empty slots, we have secondary block production mechanism based on Aura. We note that these blocks do not contribute the security of BABE meaning that best chain selection algorithm works as if Aura blocks do not exist and the randomness generation do not consider randomness in the Aura blocks.

Here, we only describe BABE together with its security properties since Aura blocks is not the part of the security of BABE.

\paragraph{BABE:}

BABE consists of \emph{epochs} ($e_1,e_2,...$) and each epoch consists of a number of sequential block production slots (\(e_i = \{sl^i_{1}, sl^i_{2},\ldots,sl^i_{t}\}\)) up to the bound  $R$.
Each validator knows in which slots that he is supposed to produce a block at the beginning of every epoch. When their slot comes, they produce the block by showing that they are assigned this slot.

The blind assignment is based on the cryptographic primitive called verifiable random function (VRF) \cite{vrf} (See Section \ref{sec:session_keys}). 
A validator in an epoch $e_m$ where $m > 2$ does the following to learn if he is eligible to produce a block in slot $sl_i^m$: He retrieves the randomness $r_{m-2}$ generated two epoch before ($e_{m-2}$). Then, he runs the VRF with the input:  randomness $r_{m-2}$ and the slot number $ sl_i^m $.  Validators in $e_1$ and $e_2$ use the randomness defined in the genesis block when they run the VRF for the slots belonging $e_1$ and $e_2$. If the random number generated by the VRF is less than the threshold $ \tau $, then the validator is the slot leader meaning that he is eligible to produce a block for this slot. We explain how to determine $\tau$ in the security analysis. 
When a validator produces a block, he adds the VRF randomness and its proof to the block showing that his VRF output is less than $\tau$  in order to convince other validators that he has a right to produce a block in the corresponding slot. The validators always generate their blocks on top of the best chain.
The best chain selection rule in BABE says that ignore the Aura blocks and select the longest chain that includes the last finalized GRANDPA block. Check Section \ref{sec:grandpa} for the details how blocks are finalized in GRANDPA. 

The randomness of an epoch $e_m$ is generated by using the blocks of the best chain that belongs to that epoch: Concatenate all  VRF values in blocks that belongs to $e_m$  (let us assume  the concatenation is \(\rho\)). Then, compute the randomness in epoch $e_{m+1}$ as $r_{m} = H(m
||\rho)$ where $ H $ is a hash function. 

Validators run periodically the relative time algorithm described below to learn the at what time a slot starts according to their local clocks..

\input{relativetime.tex}


\paragraph{- Security Overview of BABE:} Garay et al. \cite{backbone} define the properties defined below in order to obtain a secure blockchain protocol. Informally, we can describe these properties as follows:

\begin{itemize}
	\item \emph{Common Prefix Property (CP):} It ensures that the blocks which are $ k $-blocks before the last block of an honest validator's blockchain cannot be changed. We call  all unchangeable blocks  \emph{finalized} blocks. BABE satisfies CP property thanks to the honest super majority ($ \frac{2}{3}
	 $ of validators) since malicious validators are selected for a slot much less than the honest validators. It means that malicious validators does not have enough source to construct another chain which does not include one of the finalized blocks.
	\item \emph{Chain Quality (CQ):} It ensures sufficient honest block contribution to any best chain owned by an honest party.	We guarantee even in the worst case where a network delay is maximum that there will be at least one honest block in the best chain during an epoch so that the randomness cannot be biased.
	\item \emph{Chain Growth (CG):} It guarantees a minimum growth between slots. Thanks to super majority of honest validators, malicious validators cannot prevent the growth of the best chain.	
\end{itemize}
