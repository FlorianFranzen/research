\subsection{Validators (Network Maintainers)}\label{sec:validators}
 \paragraph{Keys}

 \subsubsection{NPoS mechanism for the selection of validators}

 The Polkadot network selects a new set of validators at the beginning of each new era. We denote by \nval the number of validators to be selected. This number is selected by governance and will grow linearly with the number of parachains, and is expected to be in the order of hundreds or thousands; critically, it does not need to grow with the size of the network.

A validator node should satisfy certain requirements of speed, responsiveness and security. Any DOT holder who is up to the task can submit their candidacy to become a validator. If selected, they will have their stake locked and will be economically compensated for their work, or slashed in the eventual case of a misconduct (see Section~\ref{sec:economics} for details on incentives). Additionally, any DOT holder can participate in the security of the network as a \emph{nominator}. Each nominator publishes a list (of any size) of the validator candidates that she trusts, together with an amount of DOTs that she is willing to lock to participate in NPoS. If one or more of her trusted candidates is elected during an era, she will share with them the validation rewards or slashings on a pro-rata basis. 

Unlike the case for validators, there can be an unlimited number of nominators participating in NPoS in any given era. This approach allows for a massive amount of DOTs to be staked -- and thus be subject to slashing -- which ensures a high level of security. In fact, we expect that the DOTs staked in NPoS will make a significant percentage of all available DOTs, with only a small fraction corresponding to validators' stake and the rest coming from nominators. The business model of nominators revolves around keeping track of the reputation of validators (their security practices, performance, cost efficiency, history of slashing events, etc.), and as such they play a vital role in the network. As a consequence, and unlike other PoS-based blockchains, reputation will be a more valuable asset than stake for validators.

\textbf{The validator election problem.} As a permissionless network, Polkadot elects validators via a decentralized mechanism that optimizes a series of carefully selected, fair and publicly known goals. We call this the \emph{validator election problem}, which in formal terms is a multi-winner election problem based on approval ballots, where nominators have a voting power proportional to their stake, and each nominator submits a ballot with a list of validator candidates that she supports. The solution that the mechanism outputs consists of a committee of \nval validators, together with a precise distribution of each nominator's stake among the elected validators that she backs. The goals that we optimize for are \emph{security} and \emph{fair representation}.

\textbf{Security.} 

We
We consider two objectives, both of which have been recently discussed in the literature of computational social choice. The first one is achieving the property of \emph{proportional justified representation} (PJR). The second objective, called \emph{maximin support}, is maximizing the minimum amount of stake that backs any elected validator. As we explain below, the first objective aligns with the notions of decentralization among validators as well as user satisfaction, while the second one aligns with the security level of the consensus protocol. 

\textbf{Proportional justified representation.}

